\documentclass[a4paper]{article}

%\usepackage{ctex}
\usepackage{fancyhdr}
% \usepackage{extramarks}
\usepackage{cite}
\usepackage{color}
\usepackage{float}
\usepackage{amsmath}
\usepackage{amsthm}
\usepackage{setspace}
\usepackage{amsmath}
\usepackage{amssymb}
\usepackage{cmap}
\usepackage{graphicx}
\usepackage{geometry}
\usepackage{hyperref}
\usepackage{indentfirst}
\usepackage{makecell}
\usepackage{mathrsfs}
\usepackage{multirow}
\usepackage{enumerate}
\usepackage{bm}
\usepackage{algpseudocode}
\usepackage{algorithm}
\usepackage{tikz}
\usepackage{subcaption}
\usetikzlibrary{automata,positioning,topaths}
%\usepackage{xeCJK}
%\usepackage{minted}
\title{
	\vspace{2in}
	\textbf{\courseName} \\
	\textbf{\homeworkID} \\
	\semester
	\vspace{3in}
}
\author{
	\textbf{\authorName} \\
	Student ID: \authorID
}
\date{}

\topmargin=-0.45in
\evensidemargin=0in
\oddsidemargin=0in
\textwidth=6.5in
\textheight=9.0in
\headsep=0.25in

\linespread{1.2}

\pagestyle{fancy}
\lhead{\authorName}
\chead{\courseName}
\rhead{Problem \arabic{chapterCounter}.\arabic{problemCounter}}
\cfoot{\thepage}

\newcounter{chapterCounter}
\newcounter{problemCounter}
\newcounter{partCounter}

\newenvironment{problem}[2]{
	\setcounter{chapterCounter}{#1}
	\setcounter{problemCounter}{#2}
	\section*{Problem \arabic{chapterCounter}.\arabic{problemCounter}}
	\setcounter{partCounter}{1}
}{}
\newenvironment{subproblem}[0]{
	\subsection*{Part \Alph{partCounter}}
	\stepcounter{partCounter}
}{}

\newtheorem{theorem}{Theorem}%[section]
\newtheorem{proposition}[theorem]{Proposition}
\newtheorem{lemma}[theorem]{Lemma}
\newtheorem{corollary}[theorem]{Corollary}
\newtheorem{definition}[theorem]{Definition}

\newcommand{\solution}{\paragraph{Solution}}
\newcommand{\upperbound}[1]{O\left(#1\right)}
\newcommand{\lowerbound}[1]{\Omega\left(#1\right)}
\newcommand{\exactbound}[1]{\Theta\left(#1\right)}

\newcommand{\Z}{\mathbb{Z}}
\newcommand{\R}{\mathbb{R}}
\newcommand{\Q}{\mathbb{Q}}
\newcommand{\N}{\mathbb{N}}
\newcommand{\downcast}[1]{\lfloor#1\rfloor}
\newcommand{\upcast}[1]{\lceil#1\rceil}
\newcommand{\courseName}{Deep Learning and Application}
\newcommand{\homeworkID}{Final Project}
\newcommand{\authorName}{Tianyao Chen, Runzhe Yang, Xingyuan Sun}
\newcommand{\authorID}{5140309566, 5140309562, 5140309561}
\newcommand{\semester}{2016-2017 Fall}
\begin{document}
\maketitle
\pagebreak

\section{Introduction}
\section{Data Sets}

In this project, we made $3$ datasets from the original data. They are
\begin{enumerate}
\item f0.shift.160

	This dataset is made by shift the beginning of ``f0'' feature to its first non-zero element. Then we pad $0$s at the end of the data to make its length be $160$ (since the longest data has length $157$).

	ToDo: Fix this for data.shift.json

	The dataset is saved in file ``shared data/data.f0.160.json''.
\item data.linear

	ToDo: Description of this dataset.

	The dataset is saved in file ``shared data/data\_linear.json''.

\item data.quad
	
	ToDo: Description of this dataset.

	The dataset is saved in file ``shared data/data\_quad.json''.

\end{enumerate}

\section{Models}

We tested two models on the datasets we have made.
\begin{enumerate}
\item pure.fc

	This model has just one fully connected layer.

	ToDo: Hyperparameters

\item cnn.fc

	This model has one convolution layer followed by one fully connected layer.

	ToDo: Hyperparameters
\end{enumerate}

\section{Experiments}

We tried each model on each dataset, implemented by each framework.

\subsection{Hyperparameters}

\begin{figure}[H]
\centering
\begin{tabular}{|r|c|c|c|}
\hline
 & value \\
\hline
regularization strength & $0.01$ \\
\hline
batch size & $10$ \\
\hline
optimizer & SGD \\
\hline
learning rate & $3\times10^{-5}$\\
\hline
learning rate decay & N/A \\
\hline
number of epochs & $2000$ \\
\hline
\end{tabular}
\caption{Hyperparameters settings for ``pure.fc'' model.}
\end{figure}

\begin{figure}[H]
\centering
\begin{tabular}{|r|c|c|c|}
\hline
 & value \\
\hline
regularization strength & $0.01$ \\
\hline
batch size & $10$ \\
\hline
optimizer & SGD \\
\hline
learning rate & $3\times10^{-5}$\\
\hline
learning rate decay & N/A \\
\hline
number of epochs & $2000$ \\
\hline
\end{tabular}
\caption{Hyperparameters settings for ``cnn.fc'' model.}
\end{figure}

\subsection{Experiments result}

Now we show the performance and time consumption.

\begin{figure}[H]
\centering
\begin{tabular}{|r|r|c|c|c|}
\hline
 & & data.shift & data.linear & data.quad \\
\hline
pure.fc & training set accuracy & \\
 & test set accuracy & \\
 & test\_new set accuracy & \\
 & time & \\
\hline
cnn.fc & training set accuracy & \\
 & test set accuracy & \\
 & test\_new set accuracy & \\
 & time & \\
\hline
\end{tabular}
\caption{Performance and time consumption on Torch.}
\end{figure}

\begin{figure}[H]
\centering
\begin{tabular}{|r|r|c|c|c|}
\hline
 & & data.shift & data.linear & data.quad \\
\hline
pure.fc & training set accuracy & \\
 & test set accuracy & \\
 & test\_new set accuracy & \\
 & time & \\
\hline
cnn.fc & training set accuracy & \\
 & test set accuracy & \\
 & test\_new set accuracy & \\
 & time & \\
\hline
\end{tabular}
\caption{Performance and time consumption on Theano.}
\end{figure}

\begin{figure}[H]
\centering
\begin{tabular}{|r|r|c|c|c|}
\hline
 & & data.shift & data.linear & data.quad \\
\hline
pure.fc & training set accuracy & & $91.00\%$ & $92.75\%$ \\
 & test set accuracy & & $100.00\%$ & $100.00\%$ \\
 & test\_new set accuracy & & $92.54\%$ & $92.98\%$ \\
 & time & & $35.92$s & $34.87$s \\
\hline
cnn.fc & training set accuracy & & $95.00\%$ & \\
 & test set accuracy & & $100.00\%$ & \\
 & test\_new set accuracy & & $86.40\%$ & \\
 & time & & $127.56$s & \\
\hline
\end{tabular}
\caption{Performance and time consumption on Tensorflow.}
\end{figure}

\section{Conclusion}

\end{document}